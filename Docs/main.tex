\documentclass[a4paper, 12pt]{article} % A4-Seitenformat und Schriftgröße 12pt

% Pakete für Sprache, Schrift und andere wichtige Anpassungen
\usepackage[utf8]{inputenc} % UTF-8 Unterstützung für Umlaute (ä, ü, ö)
\usepackage[T1]{fontenc}    % Korrekte Silbentrennung bei Umlauten
\usepackage[ngerman]{babel} % Deutsche Spracheinstellungen (z. B. Texttrennung, Typografie)
\usepackage{csquotes}       % Korrekte deutsche Anführungszeichen („ “)
\usepackage{hyperref} 		  % Klickbare Links im PDF
\usepackage{geometry}       % Seitenränder anpassen
\usepackage{tikz}           % Grafiken und Diagramme
\usepackage{pgfplots}			  % Diagramme und Plots

\geometry{a4paper, left=3cm, right=3cm, top=2.5cm, bottom=2.5cm} % Beispiel für Seitenränder

\title{Automatisierter Smarter Cocktail-Automat mit Cloud Anbindung}
\author{ 
  Jonas Eck \\
	Technische Hochschule Würzburg-Schweinfurt\\
	\and 
	Phillip Schön \\
	Technische Hochschule Würzburg-Schweinfurt\\
	\and 
  Constantin Thein\\
	Technische Hochschule Würzburg-Schweinfurt\\
	}

\date{\today}

% TODO: 
% - [x] Einleitung
% - [x] Technologischer Hintergrund und Grundlagen
% - [x] Anforderungen und Konzept
% - [ ] Umsetzung der Hardware
% - [x] Backend-Architektur und Implementierung
% - [ ] Mobile App-Entwicklung
% - [ ] Integration und Tests 
% - [ ] Ergebnisse und Bewertung 
% - [ ] Fazit und Ausblick 
% - [ ] Anhang
% - [ ] Literaturverzeichnis

\begin{document}

\maketitle
\newpage
\tableofcontents
\newpage

\begin{abstract}
Im Rahmen dieses Projekts wurde ein smarter Cocktailautomat entwickelt, der wesentliche Elemente des
Internet of Things (IoT) mit den Anforderungen moderner Smart-Home-Lösungen verbindet. Ziel war die
Umsetzung eines skalierbaren Prototyps, der die automatisierte Zubereitung von Cocktails ermöglicht 
und dabei durch eine intuitive Bedienung über eine Smartphone-App sowie die Anbindung an ein 
cloudbasiertes Backend besticht.

Die Lösung umfasst drei Kernkomponenten: eine modular aufgebaute Hardware für die 
Cocktailzubereitung, ein containerisiertes Backend gehostet in Google Cloud Run mit einer 
Cloud-SQL-Datenbank sowie eine mobile App, die eine benutzerfreundliche Verwaltung von Rezepten, 
Zutaten und Maschinenkonfigurationen erlaubt. Besonderes Augenmerk wurde auf die Skalierbarkeit und 
Erweiterbarkeit des Systems gelegt, wodurch sowohl mehrere Benutzer als auch mehrere Automaten 
unterstützt werden können.

Das Projekt zeigt die Potenziale von IoT-Technologien auf, alltägliche Aufgaben zu automatisieren 
und dabei innovative Ansätze für Smart-Home-Umgebungen zu entwickeln. Gleichzeitig wurden 
Herausforderungen wie die Einarbeitung in neue Technologien, der Umgang mit komplexen 
Elektronikkomponenten sowie die effiziente Integration von Cloud-Diensten gemeistert.
\end{abstract}
\newpage

\section{Einleitung}
\subsection{Motivation}
Die Idee eines smarten Cocktailautomaten entstand im Kontext unserer Erfahrungen beim Organisieren 
und Betreiben von Cocktailbars auf Fakultätsveranstaltungen. Diese praktischen Erfahrungen zeigten, 
wie zeitaufwendig die manuelle Zubereitung von Cocktails insbesondere bei hohem Gästeaufkommen sein 
kann. Mit der zunehmenden Verbreitung von Smart-Home-Geräten und IoT-Lösungen kam die Idee auf, 
diesen Prozess durch Automatisierung zu vereinfachen und so die Cocktailzubereitung effizienter und 
benutzerfreundlicher zu gestalten. 

Ein weiterer Beweggrund war die Möglichkeit, aktuelle Technologien wie Cloud-Computing und mobile 
Anwendungen in einer praxisnahen IoT-Anwendung zu integrieren und deren Potenzial in einem kreativen 
Projekt auszuloten. Während der COVID-19-Pandemie entstanden zahlreiche Do-it-yourself-Projekte im 
Bereich der Automatisierung, welche uns zu dieser Idee inspirierten.

\subsection{Zielsetzung}
Das Hauptziel des Projekts war die Entwicklung eines skalierbaren Prototyps, der die 
Cocktailzubereitung durch Automatisierung weitgehend vereinfacht. Die Bedienung des Automaten 
sollte intuitiv und nutzerfreundlich sein, sodass Benutzer per Smartphone-App mit wenigen Klicks 
einen Cocktail 'auf Knopfdruck' zubereiten können. Um eine hohe Flexibilität und Erweiterbarkeit 
zu gewährleisten, wurde eine Cloud-basierte Architektur gewählt, die es ermöglicht, mehrere 
Automaten gleichzeitig zu steuern oder eine Maschine mehreren Benutzern zugänglich zu machen. 

Ein weiterer Aspekt des Projekts war der Fokus auf Kosteneffizienz: Ziel war es, eine Lösung zu 
entwickeln, die im Vergleich zu bestehenden kommerziellen Produkten kostengünstiger ist, ohne dabei 
an Funktionalität und Skalierbarkeit einzubüßen.

\subsection{Relevanz}
Das Projekt verbindet aktuelle Trends im Bereich der Smart-Home-Technologien mit praxisorientierten 
Anwendungen des Internet of Things (IoT). Insbesondere die Integration von Cloud-Diensten und 
mobilen Applikationen in eine physische Maschine bietet ein breites Anwendungsfeld für ähnliche 
Automatisierungslösungen, die über den Cocktailautomaten hinausgehen. Durch den Prototyp wird 
gezeigt, wie IoT-Technologien genutzt werden können, um alltägliche Prozesse effizienter zu 
gestalten. 

Darüber hinaus stellt das Projekt eine wertvolle Gelegenheit dar, sich mit modernen Cloud-
Technologien wie Google Cloud Run und Cloud SQL sowie mit der Entwicklung mobiler Anwendungen und 
hardwarebezogener Softwarelösungen auseinanderzusetzen. Dies spiegelt die zunehmende Relevanz 
solcher Technologien in der Industrie wider.

\subsection{Überblick über die Arbeit}
Die vorliegende Projektarbeit gliedert sich wie folgt: Zunächst wird der technologische Hintergrund 
des Projekts vorgestellt, inklusive einer Einführung in IoT und Smart-Home-Systeme sowie der 
verwendeten Technologien. Anschließend werden die Anforderungen und das Konzept des Cocktail-
Automaten dargelegt. Die darauf folgenden Kapitel widmen sich der Umsetzung der drei zentralen 
Komponenten des Systems: der Hardware, dem containerisierten Backend sowie der mobilen App. 

Im weiteren Verlauf wird die Integration der Komponenten beschrieben und die durchgeführten Tests 
dokumentiert. Abschließend werden die erzielten Ergebnisse zusammengefasst, eine kritische Bewertung
 vorgenommen und ein Ausblick auf mögliche Weiterentwicklungen gegeben. Herausforderungen wie der 
 Umgang mit neuen Technologien, die Skalierung des Systems sowie der zeitliche Umfang des Projekts 
 werden im Rahmen der Arbeit reflektiert.

    
\newpage

\section{Technologischer Hintergrund und Grundlagen}(3–5 Seiten)
\subsection{IoT und Smart-Home-Systeme}
Im Rahmen dieses Projekts definieren wir das Internet of Things (IoT) als eine Hardwareanwendung, 
die über eine Internetverbindung steuerbar ist. Der Cocktailautomat lässt sich durch eine mobile 
App aus der Ferne bedienen, wobei besonders der Cloud-Aspekt und die Integration der Steuerung über 
das Internet hervorzuheben sind. Die Wahl eines standardisierten IoT-Protokolls wie MQTT, Zigbee 
oder BLE wurde bewusst vermieden. Stattdessen wird die Kommunikation ausschließlich über das 
HTTP-Protokoll abgewickelt, was eine einfache Implementierung und Wartung ermöglicht.

\subsection{Hardware}
Für die Hardwareplattform fiel die Wahl auf den Raspberry Pi Zero 2 W. Der Pi wurde aufgrund seiner 
einfachen Programmierbarkeit und der Verfügbarkeit relevanter Python-Bibliotheken ausgewählt. 
Zunächst war auch eine Lösung zur WLAN-Konfiguration des Raspberry Pi über die App angedacht, konnte 
jedoch aufgrund technischer Schwierigkeiten nicht im zeitlichen Rahmen der Arbeit realisiert werden. 
Der Automat nutzt einen Steppermotor, einen Linearmotor, sechs Membranpumpen, zwei Mikroendschalter, 
eine Waage sowie ein 16x2 LCD-Display.

\subsection{Backend und Cloud-Technologien}
Das Backend des Systems basiert auf Google Cloud Run, da es einfach aufzusetzen ist und zu Beginn 
ein kostenloses Guthaben von 300 US-Dollar bietet. Cloud Run ermöglicht die schnelle Bereitstellung 
von Containern, was die Skalierbarkeit und Wartbarkeit des Systems vereinfacht. Die 
Containerisierung des Backends sorgt für eine saubere Trennung der Anwendungslogik und die einfache 
Verwaltung der Software. Cloud SQL wurde als Datenbanklösung gewählt, da es eine hochverfügbare und 
ausfallsichere Lösung bietet und sich gut in die Google Cloud-Umgebung integriert. Als 
Programmiersprache wurde Go verwendet, ergänzt durch Bibliotheken wie \texttt{Mux}, um HTTP-Routen 
zu verwalten.

\subsection{Datenbank und API}
Für das Backend wurde eine relationale Datenbanklösung mit PostgreSQL genutzt, die aufgrund der 
guten Vorkenntnisse im Team und der Zuverlässigkeit der Technologie ausgewählt wurde. Die 
Tabellenmodelle für Cocktails und Zutaten wurden so entwickelt, dass eine effiziente und flexible 
Speicherung von Cocktailrezepten und Benutzerinformationen möglich ist. Als API-Technologie wurde 
eine REST-ähnliche Architektur gewählt, bei der statische Endpunkte definiert wurden, um Anfragen 
zu verarbeiten.

\subsection{Mobile App-Entwicklung}
Die mobile App wurde mit dem Framework \texttt{Flutter} entwickelt, da es die Möglichkeit bietet, 
für verschiedene Plattformen (iOS und Android) gleichzeitig zu entwickeln. Dies spart 
Entwicklungszeit und stellt sicher, dass die App auf mehreren Geräten konsistent funktioniert. 
Das Design der App orientiert sich an modernen ästhetischen Prinzipien, mit dem Ziel, eine 
benutzerfreundliche und ansprechende Oberfläche zu schaffen.

\subsection{Sicherheitsaspekte}
Sicherheitsmechanismen wurden sowohl im Backend als auch in der App implementiert. Wichtige 
API-Endpunkte sind durch API-Schlüssel abgesichert, wobei für die Kommunikation zwischen der App 
und der Hardware unterschiedliche Schlüssel verwendet werden. Zudem werden Endpunkte, die auf 
benutzerspezifische Daten zugreifen, mit JSON Web Tokens (JWT) geschützt, um sicherzustellen, 
dass nur autorisierte Benutzer auf ihre eigenen Daten zugreifen können. Die Kommunikation zwischen 
den Komponenten erfolgt über HTTPS, was eine sichere Übertragung der Daten gewährleistet.

\subsection{Weitere Technologien und Tools}
Für das Projekt wurde intensiv Docker und Docker Compose eingesetzt, um eine konsistente Umgebung 
für die Entwicklung und das Deployment zu schaffen. Dadurch wird sichergestellt, dass die Anwendung 
sowohl lokal als auch in der Cloud identisch läuft, was die Entwicklung und Wartung des Systems 
erheblich vereinfacht.

% \subsection{Zusammenfassung} % TODO: FIND A BETTER TITLE

% Insgesamt zeigt dieses Projekt, wie moderne Cloud- und IoT-Technologien in einem smarten System 
% kombiniert werden können, um eine benutzerfreundliche und skalierbare Lösung zu schaffen. Trotz der 
% Herausforderungen bei der Implementierung von WLAN-Konfigurationsfunktionen und der Integration 
% komplexer IoT-Protokolle konnte ein funktionierendes System entwickelt werden, das durch einfache 
% HTTP-Kommunikation zwischen den Komponenten und einer stabilen Cloud-Infrastruktur unterstützt wird.    
\newpage

\section{Anforderungen und Konzept}(4–5 Seiten)
\section{Anforderungen und Konzept}
% Hier folgt der Inhalt des Abschnitts "Anforderungen und Konzept".
    
\newpage

\section{Umsetzung der Hardware}(4–6 Seiten)
\subsection{Überblick}
Die Hardwareplattform für das Projekt bildet das Herzstück des Cocktailautomaten und wurde mit dem Ziel ausgewählt, eine robuste und flexible Steuerung der mechanischen und elektronischen Komponenten zu gewährleisten. Die Wahl fiel auf den \textbf{Raspberry Pi Zero 2 W}, eine kompakte und leistungsfähige Single-Board-Computer-Lösung, die sich aufgrund ihrer Vielseitigkeit und der umfangreichen Unterstützung von Python-Bibliotheken ideal für IoT-Anwendungen eignet. Ergänzt wird der Raspberry Pi durch eine Vielzahl von Sensoren, Aktoren und Peripheriegeräten, die zusammen die präzise Steuerung des Automaten ermöglichen.

\subsection{Zentrale Steuerung}
Der Raspberry Pi Zero 2 W übernimmt die zentrale Steuerung des Systems. Mit einem Quad-Core ARM Cortex-A53-Prozessor und integrierter WLAN- und Bluetooth-Konnektivität eignet sich der Pi optimal für die Verarbeitung der Steuerungslogik und die Kommunikation mit dem Backend. Seine kompakte Bauform erleichtert die Integration in das Gehäuse des Automaten.

\subsection{Mechanische Komponenten}
\subsubsection{Schrittmotor für die Positionierung}
Ein Schrittmotor steuert die präzise Bewegung eines Förderbandsystems, das die verschiedenen Getränkezutaten in die richtige Position bringt. Mit einem Dir- und Pul-Pin-Design wird der Schrittmotor über einen \textbf{Treiber} angesteuert, der eine feine Steuerung der Bewegung und Geschwindigkeit ermöglicht. Ein Schlitten bewegt sich zwischen den Slots, die durch Endschalter (Limit Switches) abgegrenzt sind.

\subsubsection{Linearantrieb für alkoholische Getränke}
Ein Linearmotor aktiviert die Dosiermechanismen für alkoholische Zutaten, die kopfüber montiert sind. Der Antrieb hebt die Dosiermechanik an, um Flüssigkeiten aus den Flaschen freizugeben, und stellt sicher, dass die gewünschte Menge präzise abgefüllt wird.

\subsubsection{Membranpumpen für nicht-alkoholische Zutaten}
Sechs Membranpumpen sind für die Abgabe von nicht-alkoholischen Zutaten zuständig. Diese Pumpen ermöglichen eine genaue Dosierung der Flüssigkeiten und werden einzeln über GPIO-Pins des Raspberry Pi angesteuert. Jede Pumpe ist einem festen Slot zugewiesen.

\subsection{Sensorik}
\subsubsection{Endschalter für die Positionserkennung}
Sieben Endschalter (Limit Switches) überwachen die genaue Position des Schlittenmechanismus. Ein spezieller Endschalter ist für die Kalibrierung vorgesehen und dient als Referenzpunkt, während die übrigen Schalter die Position der einzelnen Slots (1--11) abbilden.

\subsubsection{Waage für Gewichtsmessung}
Eine elektronische Waage auf Basis eines HX711-Sensors misst das Gewicht der Getränke während der Zubereitung. Sie dient nicht nur der Überwachung der genauen Flüssigkeitsmengen, sondern auch der Erkennung, ob eine Zutat aufgebraucht ist.

\subsubsection{16x2 LCD-Display}
Das LCD-Display zeigt Statusinformationen wie Fortschritt, Fehler oder Benutzerhinweise an. Es wird über das I2C-Protokoll mit dem Raspberry Pi verbunden und kann in Echtzeit aktualisiert werden.

\subsection{Stromversorgung}
Die gesamte Hardware wird von einem \textbf{24V-Netzteil} mit ausreichender Leistung (mindestens 3A) versorgt. Ein \textbf{Step-Down-Spannungsregler} wandelt die Spannung auf 5V für den Raspberry Pi und weitere elektronische Komponenten um. Zur Vermeidung von Spannungsschwankungen und elektromagnetischen Störungen wurden \textbf{Kondensatoren} integriert, die die Stabilität der Stromversorgung sicherstellen.

\subsection{Zusammenarbeit der Komponenten}
Alle Hardwarekomponenten arbeiten nahtlos zusammen, um die Zubereitung eines Cocktails zu automatisieren:
\begin{itemize}
    \item Die \textbf{Positionserkennung} sorgt dafür, dass der Schlitten immer die korrekte Position ansteuert.
    \item Die \textbf{Waage} und die Pumpen oder der Linearantrieb arbeiten zusammen, um die exakte Menge jeder Zutat abzugeben.
    \item Das \textbf{LCD-Display} liefert dem Benutzer kontinuierlich Rückmeldungen über den Fortschritt der Zubereitung.
\end{itemize}

\subsection{Integration und Herausforderungen}
Die Integration der verschiedenen Hardwarekomponenten erforderte eine sorgfältige Planung, insbesondere bei der Belegung der GPIO-Pins und der Kalibrierung der Sensoren. Eine besondere Herausforderung stellte die Stabilität der Stromversorgung dar, insbesondere bei Spitzenlasten durch den Schrittmotor oder die Pumpen. Durch die Verwendung eines ausreichend dimensionierten Netzteils und der Optimierung des Codes konnte dieses Problem erfolgreich gelöst werden.
    
\newpage

\section{Backend-Architektur und Implementierung}(4–6 Seiten)
\section{Backend-Architektur und Implementierung}
% Hier folgt der Inhalt des Abschnitts "Backend-Architektur und Implementierung".
    
\newpage

\section{Mobile App-Entwicklung}(4–5 Seiten)
% Hier folgt der Inhalt des Abschnitts "Mobile App-Entwicklung".
    
\newpage

\section{Integration und Tests}(4–5 Seiten)
\section{Integration und Tests}
% Hier folgt der Inhalt des Abschnitts "Integration und Tests".
    
\newpage

\section{Ergebnisse und Bewertung}(3–4 Seiten)
% Hier folgt der Inhalt des Abschnitts "Ergebnisse und Bewertung".

\subsection{Lasttest Ergebnisse - k6}
% Wir haben einen Lasttest durchgeführt, um die Leistung und Stabilität der API zu bewerten. Die 
% Ergebnisse zeigen, dass die API insgesamt 37.900 Anfragen verarbeiten konnte, wobei 17.097 Anfragen 
% fehlgeschlagen sind. Die durchschnittliche Antwortzeit betrug 66,26 ms, was akzeptabel ist. Die 
% Erfolgsquote für die Benutzerregistrierung und Rezeptverwaltung lag bei 9.

\subsection{Beispiele Plots und Diagramme}

\begin{tikzpicture}[scale=1]
    \begin{axis}[
        axis lines=middle,
        xlabel=,
        ylabel={},
        legend style={
            fill=pag, draw=pag!60, % Red background with black border
            font=\small, % Smaller font size
            inner sep=2pt, % Inner spacing (padding around the text)
            outer sep=1pt  % Outer spacing (margin around the border)
        },
        domain=0:8000,
        xtick=\empty,
        ytick=\empty,
        grid=both,
        grid style={line width=.1pt, draw=gray!10},
        major grid style={line width=.2pt,draw=gray!50},
        minor tick num=5,
        width=1.15*\textwidth,
        height=5cm,
        clip=false,
        ticklabel style={font=\tiny,fill=white},
        xlabel style={at={(ticklabel* cs:1)},anchor=north west},
        ylabel style={at={(ticklabel* cs:1)},anchor=south west},
        ]

        \addplot[color=red1,ultra thick,samples=100]{x};
        \addplot[color=red2,ultra thick,samples=100]{1/x};
        \addplot[color=red3,ultra thick,samples=100]{1/x^2};

        \legend{100, 300, 1000}
    \end{axis}
    \node at (3.5,-0.3) {Velocity};
    \node[rotate=90] at (-0.3,1.65) {Probability};
\end{tikzpicture}

\begin{tikzpicture}
	\begin{axis}
	\addplot3[
			surf,
	]
	{exp(-x^2-y^2)*x};
	\end{axis}
	\end{tikzpicture}

	Plotting from data:

\begin{tikzpicture}
\begin{axis}[
    title={Temperature dependence of CuSO\(_4\cdot\)5H\(_2\)O solubility},
    xlabel={Temperature [\textcelsius]},
    ylabel={Solubility [g per 100 g water]},
    xmin=0, xmax=100,
    ymin=0, ymax=120,
    xtick={0,20,40,60,80,100},
    ytick={0,20,40,60,80,100,120},
    legend style={
        fill=pag, draw=pag!60, % Red background with black border
        font=\small, % Smaller font size
        inner sep=2pt, % Inner spacing (padding around the text)
        outer sep=1pt  % Outer spacing (margin around the border)
    },
    legend pos=north west,
    ymajorgrids=true,
    grid style=dashed,
]

\addplot[color=red2, ultra thick, mark=triangle*]
    coordinates {
    (0,23.1)(10,27.5)(20,32)(30,37.8)(40,44.6)(60,61.8)(80,83.8)(100,114)
    };
    \legend{CuSO\(_4\cdot\)5H\(_2\)O}
    
\end{axis}
\end{tikzpicture}

\paragraph*{Info}
Find more examples here: \url{https://www.overleaf.com/learn/latex/Pgfplots_package}
Or here: \url{https://pgfplots.net/}


\section*{Testergebnisse}

\subsection*{Wichtige Metriken}
\begin{itemize}
    \item \textbf{Gesamtanfragen:} 37.900
    \item \textbf{Fehlerrate:} 45,11\% (17.097 Anfragen fehlgeschlagen)
    \item \textbf{Durchschnittliche Antwortzeit:} 66,26 ms
    \item \textbf{Erfolgsquote Registrierung:} 9\% (927 von 9.475 Iterationen erfolgreich)
    \item \textbf{Erfolgsquote Rezeptverwaltung:} 9\% (926 von 9.475 Iterationen erfolgreich)
\end{itemize}

\subsection*{Antwortzeiten}
\begin{figure}
    \centering
    \begin{tikzpicture}
        \begin{axis}[
            width=\textwidth,
            height=8cm,
            xlabel={Antwortzeit (ms)},
            ylabel={Prozentsatz der Anfragen},
            ymin=0, ymax=100,
            xmin=0, xmax=700,
            xtick={0,100,200,300,400,500,600,700},
            ytick={0,20,40,60,80,100},
            legend style={
                fill=pag, draw=pag!60, % Red background with black border
                font=\small, % Smaller font size
                inner sep=2pt, % Inner spacing (padding around the text)
                outer sep=1pt  % Outer spacing (margin around the border)
            },
            legend pos=north east,
            grid=both,
            ]
            \addplot[red3,ultra thick] coordinates {
                (14, 0) (87, 50) (105, 90) (140, 95) (522, 99) (644, 100)
            };
            \addlegendentry{Antwortzeit}
        \end{axis}
    \end{tikzpicture}
    \caption{Antwortzeiten der API (90. Perzentil: 105 ms, 95. Perzentil: 140 ms)}
    \label{fig:response_time}
\end{figure}

\subsection*{Erfolgsquoten der Checks}
\begin{figure}
    \centering
    \begin{tikzpicture}
        \begin{axis}[
            ybar,
            width=\textwidth,
            height=8cm,
            xlabel={API-Endpunkt},
            ylabel={Erfolgsquote (\%)},
            ymin=0, ymax=100,
            symbolic x coords={register, login, create drink, create recipe},
            xtick=data,
            bar width=0.5cm,
            nodes near coords,
            nodes near coords align={vertical},
            ]
            \addplot coordinates {(register, 9) (login, 100) (create drink, 100) (create recipe, 9)};
        \end{axis}
    \end{tikzpicture}
    \caption{Erfolgsquoten der Haupt-API-Endpunkte}
    \label{fig:success_rates}
\end{figure}

\subsection*{Fehlerrate über die Zeit}
\begin{figure}
    \centering
    \begin{tikzpicture}
        \begin{axis}[
            width=\textwidth,
            height=8cm,
            xlabel={Zeit (Minuten)},
            ylabel={Fehlerrate (\%)},
            xmin=0, xmax=20,
            ymin=0, ymax=100,
            grid=both,
            ]
            \addplot[red, thick] coordinates {
                (0, 30) (5, 45) (10, 50) (15, 55) (20, 45)
            };
            \addlegendentry{Fehlerrate}
        \end{axis}
    \end{tikzpicture}
    \caption{Fehlerrate im Verlauf des Tests}
    \label{fig:error_rate}
\end{figure}

\section*{Schlussfolgerungen}
Die Ergebnisse zeigen, dass:
\begin{itemize}
    \item Die Benutzerregistrierung und Rezeptverwaltung besonders fehleranfällig sind (nur 9\% Erfolgsquote).
    \item Die API insgesamt eine hohe Fehlerrate von 45,11\% aufweist, was auf Skalierungs- oder Ressourcenprobleme hindeuten könnte.
    \item Die durchschnittliche Antwortzeit (66,26 ms) akzeptabel ist, aber in Spitzenzeiten (max. 644 ms) stark ansteigt.
\end{itemize}

\section*{Empfehlungen}
\begin{itemize}
    \item Optimieren Sie die Ressourcen- und Skalierungseinstellungen von Google Cloud Run.
    \item Reduzieren Sie die Last in der Benutzerregistrierung durch bessere Datenbankabfragen oder asynchrone Prozesse.
    \item Testen Sie kleinere Benutzergruppen, um die Schwellenwerte für Stabilitätsprobleme zu identifizieren.
\end{itemize}
\newpage

\section{Fazit und Ausblick}(2 Seiten)
% Hier folgt der Inhalt des Abschnitts "Fazit und Ausblick".
\subsection{Fazit}
Das Projekt hat gezeigt, dass die Kombination aus Hardware, Software und Cloud-Diensten 
erfolgreich zur Automatisierung der Cocktailzubereitung genutzt werden kann. Die definierten 
Hauptziele des Projekts wurden weitestgehend erreicht:

\begin{itemize}
    \item Die Maschine kann Cocktails entsprechend der über die App ausgewählten Rezepte 
        zusammenmixen.
    \item In der App können Benutzer Getränke, Zutaten und Slots anlegen, ändern und löschen.
    \item Die Integration zwischen App, Backend und Hardware funktioniert stabil und erfüllt die 
        Kernanforderungen.
\end{itemize}

Besonders positiv hervorzuheben ist die erfolgreiche Umsetzung folgender Punkte:

\begin{itemize}
    \item Hosting des Backends über Google Cloud
    \item Erstellung einer funktionsfähigen Android-App
    \item Zuverlässige Verkabelung und Integration der Hardware
\end{itemize}
Trotz der Erfolge gab es auch Herausforderungen, die nur teilweise gelöst werden konnten. 
Beispielsweise verursachen die Treiber des Schrittmotors und des Linearmotors Schwankungen im 
Ground, was die Übertragung von Daten beeinträchtigt. Dies führt insbesondere bei feineren 
Komponenten wie der Waage zu Störungen, obwohl diese außerhalb des Systems problemlos funktioniert 
hat.

Zudem gab es Einschränkungen bei der Auswahl bestimmter Hardware-Komponenten. So sind die 
Spirituosenausschänker gelegentlich unzuverlässig, was zu Ausfällen führen kann. Die Kraft, die 
erforderlich ist, um diese Ausschänker zu bedienen, wurde ebenfalls unterschätzt. Dies machte 
einen Wechsel von Servomotoren zu einem leistungsfähigeren Linearmotor notwendig.

Insgesamt hat das Projekt die Kenntnisse im Bereich Elektronik, Stromversorgung sowie in der 
Entwicklung und Bereitstellung von Apps und Backends erheblich erweitert.

\subsection{Ausblick}
Für die Zukunft sind mehrere Erweiterungen und Optimierungen geplant, um die Funktionalität und 
Zuverlässigkeit des Systems weiter zu verbessern:

\begin{itemize}
    \item \textbf{Ansteuerbare LEDs:} Diese sollen nicht nur für eine ansprechende Optik sorgen, sondern
    auch leere Flaschen in den Slots durch die gezielte Ansteuerung einzelner LEDs sichtbar machen. Die Umsetzung wird jedoch durch Störsignale im Ground beeinträchtigt, die zunächst behoben werden müssen.
    \item \textbf{Rollenzuweisung:} Benutzer sollen unterschiedliche Rollen erhalten, um 
        beispielsweise das Verändern von Rezepten und Zutaten auf autorisierte Nutzer zu beschränken.
    \item \textbf{Hardwareverwaltung in der App:} Es soll möglich sein, mehrere Smartender-Geräte 
        über die App zu steuern und zu verwalten.
    \item \textbf{Spülprogramm:} Ein automatisches Spülprogramm soll die Reinigung der Schläuche 
        erleichtern und Restflüssigkeiten entfernen.
    \item \textbf{WLAN-Einrichtung:} Die WLAN-Konfiguration soll über die App erfolgen, um die 
        Notwendigkeit einer Tastatur und eines Displays am Raspberry Pi zu vermeiden und die 
        Einrichtung zu erleichtern.
    \item \textbf{Integration einer Waage:} Eine Waage soll die genaue Bestimmung der im Becher 
        befindlichen Flüssigkeitsmenge ermöglichen. Dies erfordert jedoch die Behebung der 
        bestehenden Störsignale im Ground.
\end{itemize}
Diese Erweiterungen würden den Smartender zu einer noch vielseitigeren Lösung für private und 
kommerzielle Anwendungen machen.    
\newpage

\section{Anhang}(variabel)
% Hier folgt der Inhalt des Abschnitts "Anhang".
    
\newpage

\section{Literaturverzeichnis}(1–2 Seiten)
% Hier folgt der Inhalt des Abschnitts "Literaturverzeichnis".
\begin{thebibliography}{9}

  \bibitem[Doe]{doe} \emph{First and last \LaTeX{} example.},
  John Doe 50 B.C. 

  \bibitem[RaspberryPi]{raspberrypi} \emph{Raspberry Pi Zero 2 W Product Brief.},
  Raspberry Pi Foundation, 2021. Available: \url{https://datasheets.raspberrypi.com/rpizero2/raspberry-pi-zero-2-w-product-brief.pdf}

  \bibitem[Go Language]{golang} \emph{The Go Programming Language.},
  Go Language, 2025, Available: \url{https://go.dev}

  \bibitem[Google]{googlecloudrun} \emph{Google Cloud Run.},
  Google Cloud, 2025, Available: \url{https://cloud.google.com/run}

  \bibitem[Google]{googleartifactregistry} \emph{Google Artifact Registry.},
  Google Cloud, 2025, Available: \url{https://cloud.google.com/artifact-registry}

  \bibitem[Google]{googlecloudsql} \emph{Google Cloud SQL.},
  Google Cloud, 2025, Available: \url{https://cloud.google.com/sql}

  \bibtex[Google]{googlecloudlogging} \emph{Google Cloud Logging.},
  Google Cloud, 2025, Available: \url{https://cloud.google.com/logging}
  
\end{thebibliography}    
\newpage

\end{document}
