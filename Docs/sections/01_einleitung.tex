\subsection{Motivation}
Die Idee eines smarten Cocktailautomaten entstand im Kontext unserer Erfahrungen beim Organisieren 
und Betreiben von Cocktailbars auf Fakultätsveranstaltungen. Diese praktischen Erfahrungen zeigten, 
wie zeitaufwendig die manuelle Zubereitung von Cocktails insbesondere bei hohem Gästeaufkommen sein 
kann. Mit der zunehmenden Verbreitung von Smart-Home-Geräten und IoT - Lösungen kam die Idee auf, 
diesen Prozess durch Automatisierung zu vereinfachen und so die Cocktailzubereitung effizienter und 
benutzerfreundlicher zu gestalten. 
\\
Ein weiterer Beweggrund war die Möglichkeit, aktuelle Technologien wie Cloud-Computing und mobile 
Anwendungen in einer praxisnahen IoT - Anwendung zu integrieren und deren Potenzial in einem kreativen 
Projekt auszuloten. Während der COVID-19 - Pandemie entstanden zahlreiche Do-it-yourself-Projekte im 
Bereich der Automatisierung, welche uns zu dieser Idee inspirierten.

\subsection{Zielsetzung}
Das Hauptziel des Projekts war die Entwicklung eines skalierbaren Prototyps, der die 
Cocktailzubereitung durch Automatisierung weitgehend vereinfacht. Die Bedienung des Automaten 
sollte intuitiv und nutzerfreundlich sein, sodass Benutzer per Smartphone-App mit wenigen Klicks 
einen Cocktail 'auf Knopfdruck' zubereiten können. Um eine hohe Flexibilität und Erweiterbarkeit 
zu gewährleisten, wurde eine Cloud - basierte Architektur gewählt, die es ermöglicht, mehrere 
Automaten gleichzeitig zu steuern oder eine Maschine mehreren Benutzern zugänglich zu machen. 

Ein weiterer Aspekt des Projekts war der Fokus auf Kosteneffizienz: Ziel war es, eine Lösung zu 
entwickeln, die im Vergleich zu bestehenden kommerziellen Produkten kostengünstiger ist, ohne dabei 
an Funktionalität und Skalierbarkeit einzubüßen.

\subsection{Relevanz}
Das Projekt verbindet aktuelle Trends im Bereich der Smart-Home-Technologien mit praxisorientierten 
Anwendungen des Internet of Things (IoT). Insbesondere die Integration von Cloud-Diensten und 
mobilen Applikationen in eine physische Maschine bietet ein breites Anwendungsfeld für ähnliche 
Automatisierungslösungen, die über den Cocktailautomaten hinausgehen. Durch den Prototyp wird 
gezeigt, wie IoT-Technologien genutzt werden können, um alltägliche Prozesse effizienter zu 
gestalten. 

Darüber hinaus stellt das Projekt eine wertvolle Gelegenheit dar, sich mit modernen Cloud-
Technologien wie Google Cloud Run und Cloud SQL sowie mit der Entwicklung mobiler Anwendungen und 
hardwarebezogener Softwarelösungen auseinanderzusetzen. Dies spiegelt die zunehmende Relevanz 
solcher Technologien in der Industrie wider.

\subsection{Überblick über die Arbeit}
Die vorliegende Projektarbeit gliedert sich wie folgt: Zunächst wird der technologische Hintergrund 
des Projekts vorgestellt, inklusive einer Einführung in IoT und Smart-Home-Systeme sowie der 
verwendeten Technologien. Anschließend werden die Anforderungen und das Konzept des Cocktail-
Automaten dargelegt. Die darauf folgenden Kapitel widmen sich der Umsetzung der drei zentralen 
Komponenten des Systems: der Hardware, dem containerisierten Backend sowie der mobilen App. 

Im weiteren Verlauf wird die Integration der Komponenten beschrieben und die durchgeführten Tests 
dokumentiert. Abschließend werden die erzielten Ergebnisse zusammengefasst, eine kritische Bewertung
 vorgenommen und ein Ausblick auf mögliche Weiterentwicklungen gegeben. Herausforderungen wie der 
 Umgang mit neuen Technologien, die Skalierung des Systems sowie der zeitliche Umfang des Projekts 
 werden im Rahmen der Arbeit reflektiert.

