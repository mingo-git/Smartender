\subsection{Überblick}
Die Hardwareplattform für das Projekt bildet das Herzstück des Cocktailautomaten und wurde mit dem Ziel ausgewählt, eine robuste und flexible Steuerung der mechanischen und elektronischen Komponenten zu gewährleisten. Die Wahl fiel auf den \textbf{Raspberry Pi Zero 2 W}, eine kompakte und leistungsfähige Single-Board-Computer-Lösung, die sich aufgrund ihrer Vielseitigkeit und der umfangreichen Unterstützung von Python-Bibliotheken ideal für IoT-Anwendungen eignet. Ergänzt wird der Raspberry Pi durch eine Vielzahl von Sensoren, Aktoren und Peripheriegeräten, die zusammen die präzise Steuerung des Automaten ermöglichen.

\subsection{Zentrale Steuerung}
Der Raspberry Pi Zero 2 W übernimmt die zentrale Steuerung des Systems. Mit einem Quad-Core ARM Cortex-A53-Prozessor und integrierter WLAN- und Bluetooth-Konnektivität \cite{raspberrypi} eignet sich der Pi optimal für die Verarbeitung der Steuerungslogik und die Kommunikation mit dem Backend.

\subsection{Mechanische Komponenten}
\subsubsection{Schrittmotor für die Positionierung}
Ein Schrittmotor steuert die präzise Bewegung eines Förderbandsystems, welches die verschiedenen Getränkezutaten anfahren kann. Mit einem Dir- und Pul-Pin-Design wird der Schrittmotor über einen \textbf{Treiber} angesteuert, der eine feine Steuerung der Bewegung und Geschwindigkeit ermöglicht. Ein Schlitten bewegt sich zwischen den Slots, die durch Endschalter (Limit Switches) abgegrenzt sind.

\subsubsection{Linearantrieb für alkoholische Getränke}
Ein Linearmotor aktiviert die Dosiermechanismen für alkoholische Zutaten, die kopfüber montiert sind. Der Antrieb hebt die Dosiermechanik an, um Flüssigkeiten aus den Flaschen freizugeben, und stellt sicher, dass die gewünschte Menge präzise abgefüllt wird.

\subsubsection{Membranpumpen für nicht-alkoholische Zutaten}
Sechs Membranpumpen sind für die Abgabe von nicht-alkoholischen Zutaten zuständig. Diese Pumpen ermöglichen eine genaue Dosierung der Flüssigkeiten und werden einzeln über GPIO-Pins des Raspberry Pi angesteuert. Jede Pumpe ist einem festen Slot zugewiesen.

\subsection{Sensorik}
\subsubsection{Endschalter für die Positionserkennung}
Sechs Endschalter (Limit Switches) überwachen die genaue Position des Schlittenmechanismus. Endschalter 0 ist für die Kalibrierung vorgesehen und dient als Referenzpunkt. Außerdem werden an diesem Punkt die nicht-alkoholischen Getränke am Ende der Zubereitung zugefügt. Die übrigen Schalter (1--5) bilden die Position der Slots für alkoholische Getränke (ebenfalls 1--5) ab.

\subsubsection{Waage für Gewichtsmessung}
Eine elektronische Waage auf Basis eines HX711-Sensors misst das Gewicht der Getränke während der Zubereitung. Sie dient nicht nur der Überwachung der genauen Flüssigkeitsmengen, sondern auch der Erkennung, ob eine Zutat aufgebraucht ist.

\subsection{Stromversorgung}
Die gesamte Hardware wird von einem \textbf{24V-Netzteil} mit ausreichender Leistung (mindestens 3A) versorgt. Zwei \textbf{Step-Down-Spannungsregler} wandeln die Spannung auf 5V respektive 12V für den Raspberry Pi und weitere elektronische Komponenten um. Zur Vermeidung von Spannungsschwankungen und elektromagnetischen Störungen wurden \textbf{Kondensatoren} integriert, die die Stabilität der Stromversorgung sicherstellen.

\subsection{Zusammenarbeit der Komponenten}
Alle Hardwarekomponenten arbeiten nahtlos zusammen, um die Zubereitung eines Cocktails zu automatisieren:
\begin{itemize}
    \item Die \textbf{Positionserkennung} sorgt dafür, dass der Schlitten immer die korrekte Position ansteuert.
    \item Die \textbf{Waage} und die Pumpen oder der Linearantrieb arbeiten zusammen, um die exakte Menge jeder Zutat abzugeben.
\end{itemize}

\subsection{Pinbelegung und Schaltplan}
Die genaue Pinbelegung \ref{fig:pinning} sowie der vollständige Schaltplan \ref{fig:schaltplan} der Hardware sind im Anhang zu finden. Das Pinning-Sheet dokumentiert die Zuordnung der GPIO-Pins des Raspberry Pi zu den verschiedenen Komponenten wie Schrittmotor, Linearantrieb, Pumpen, Endschaltern und Waage. Der Schaltplan bietet einen Überblick über die elektrische Verkabelung und die Integration der Stromversorgung sowie der Steuereinheiten. Beide Dokumente sind essenziell für den Nachbau oder die Wartung des Systems.

\subsection{Integration und Herausforderungen}
Die Integration der verschiedenen Hardwarekomponenten erforderte eine sorgfältige Planung, insbesondere bei der Belegung der GPIO-Pins und der Kalibrierung der Sensoren. Eine besondere Herausforderung stellte die Stabilität der Stromversorgung dar, insbesondere bei Spitzenlasten durch den Schrittmotor oder die Pumpen. Durch die Verwendung eines ausreichend dimensionierten Netzteils und der Optimierung des Codes konnte dieses Problem erfolgreich gelöst werden.
