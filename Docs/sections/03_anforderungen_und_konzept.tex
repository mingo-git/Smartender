% Hier folgt der Inhalt des Abschnitts "Anforderungen und Konzept".
Im Rahmen des Projekts „Smarter Cocktailautomat“ wurden grundlegende Anforderungen definiert, die 
sich aus den Zielsetzungen und den zu erwartenden Einsatzszenarien ableiten lassen. Dieser 
Abschnitt gliedert sich in eine Analyse der funktionalen und nicht-funktionalen Anforderungen 
sowie die Entwicklung eines konzeptionellen Architekturansatzes, der die Interaktionen zwischen 
Hardware, Backend und mobiler App beschreibt.

\subsection{Anforderungsanalyse}

Die Anforderungsanalyse bildet die Grundlage für die Entwicklung des Prototyps. Dabei werden 
sowohl funktionale als auch nicht-funktionale Anforderungen betrachtet.

\subsection{Funktionale Anforderungen}

Funktionale Anforderungen beschreiben die Kernfunktionen, die das System erfüllen muss, um die 
Projektziele zu erreichen:

\begin{itemize}
	  \item \textbf{Benutzerregistrierung und -verwaltung:} Möglichkeit, Benutzerkonten anzulegen, zu bearbeiten und zu löschen.
	  \item \textbf{Rezeptverwaltung:} Hinzufügen, Bearbeiten und Löschen von Cocktailrezepten über die mobile App.
	  \item \textbf{Zutatenverwaltung:} Verwaltung der im Automaten verfügbaren Zutaten mit Anzeige von Füllständen.
	  \item \textbf{Automatisierte Zubereitung:} Präzise Dosierung und Mischung der ausgewählten Zutaten basierend auf den Rezepten.
	  \item \textbf{App-Steuerung:} Steuerung des Cocktailautomaten über eine intuitive Benutzeroberfläche der mobilen App.
	  \item \textbf{Synchronisation mit der Cloud:} Speicherung und Abruf von Benutzerdaten, Rezepten und Maschinenkonfigurationen über ein cloudbasiertes Backend.
\end{itemize}

\subsection{Nicht-funktionale Anforderungen}

Neben den funktionalen Aspekten spielen auch nicht-funktionale Anforderungen eine zentrale Rolle, 
um die Qualität und Benutzerfreundlichkeit des Systems sicherzustellen:

\begin{itemize}
	  \item	\textbf{Skalierbarkeit:} Das System muss in der Lage sein, mehrere Benutzer und Automaten zu unterstützen, ohne Leistungseinbußen zu erleiden.
	  \item	\textbf{Sicherheitsaspekte:} Schutz von Benutzerdaten durch verschlüsselte Kommunikation und sichere Authentifizierungsverfahren.
	  \item	\textbf{Benutzerfreundlichkeit:} Intuitive Bedienung der mobilen App sowie einfache Wartung und Erweiterung der Hardware.
	  \item	\textbf{Fehlerresistenz:} Robustheit des Systems gegenüber Hardware- und Softwarefehlern.
	  \item	\textbf{Performance:} Minimierung von Latenzen bei der Kommunikation zwischen App, Backend und Hardware.
\end{itemize}

\subsection{Konzeptionelle Architektur}

Auf Basis der definierten Anforderungen wurde eine konzeptionelle Architektur entwickelt, die das 
Gesamtsystem in seinen zentralen Komponenten beschreibt.

Gesamtsystem-Übersicht

Das System besteht aus drei Kernkomponenten:

\begin{enumerate}
  \item \textbf{Hardware:} Der Cocktailautomat als physisches Gerät übernimmt die Dosierung und Mischung der Zutaten.
  \item \textbf{Backend:} Ein containerisiertes Backend, das in Google Cloud Run gehostet wird, dient als zentrale Schnittstelle für die Speicherung und Verarbeitung von Daten.
  \item \textbf{Mobile App:} Eine Smartphone-App ermöglicht die Benutzerinteraktion mit dem System und bietet Funktionen wie Rezeptverwaltung und Steuerung des Automaten.
\end{enumerate}

Die Interaktionen zwischen diesen Komponenten sind in Abbildung \ref{fig:system_overview} 
dargestellt. Dabei wird insbesondere der Datenfluss zwischen den Modulen hervorgehoben.

\vspace{0.5 cm}
\begin{figure}[htbp]
  \centering
	\begin{tikzpicture}[
		>=stealth,   					% Standardpfeilspitzen
		outer sep=6pt,				% Abstand zwischen Text und Rahmen
		inner sep=5pt,				% Abstand zwischen Text und Rahmen
		node distance=6cm,	% Abstand zwischen den Knoten
		font=\sffamily				% Verwendete Schriftart
	]
	
	% Knoten
	\node[draw, rectangle, rounded corners, fill=red3, minimum width=2.5cm, minimum height=1.2cm] (hardware) {Hardware};
	\node[draw, rectangle, rounded corners, fill=red3, minimum width=2.5cm, minimum height=1.2cm, left of=hardware] (backend) {Backend};
	\node[draw, rectangle, rounded corners, fill=red3, minimum width=2.5cm, minimum height=1.2cm, left of=backend] (app) {Mobile App};
	
	% Pfeil 1: ein Wort pro Zeile, zentriert
	\draw[<-, thick] 
	(hardware) 
	-- node[below]{
		\parbox{2.7cm}{\centering
			Rezeptdaten,\\
			Slotbestückung 
		}
	} 
	(backend);

	% Pfeil 2: ein Wort pro Zeile, zentriert
	\draw[<->, thick] 
	(app) 
	-- node[below]{
		\parbox{2.7cm}{\centering
			Benutzerdaten,\\
			Rezepte \&\\
			Konfiguration
		}
	} 
	(backend);

	\end{tikzpicture}

	\caption{Systemübersicht: Interaktionen zwischen Hardware, Backend und mobiler App}
	\label{fig:system_overview}
\end{figure}

\paragraph*{Datenflüsse und Interaktionen}

\begin{itemize}
  \item Die Hardware kommuniziert mit dem Backend, um die aktuellen Maschinenzustände zu synchronisieren.
  \item Die mobile App greift auf das Backend zu, um Benutzerdaten, Rezepte und Maschinenkonfigurationen zu laden und zu speichern.
  \item Das Backend dient als zentrale Vermittlungsstelle und verarbeitet sowohl Steuerbefehle von der App als auch Statusmeldungen der Hardware.
\end{itemize}

\subsection{Schwerpunktsetzung}

Im Rahmen des Projekts wurde besonderes Augenmerk auf die folgenden Aspekte gelegt:

\begin{itemize}
  \item \textbf{Hardware:} Die Hardwareentwicklung zielte auf eine modulare und erweiterbare Bauweise ab, um zukünftige Anpassungen und Erweiterungen zu ermöglichen.
  \item \textbf{Backend:} Aufgrund der hohen Anforderungen an Skalierbarkeit und Sicherheit wurde das Backend so konzipiert, dass es sowohl eine zuverlässige Kommunikation zwischen den Komponenten als auch eine effiziente Datenverwaltung ermöglicht.
  \item \textbf{Mobile App:} Der Fokus lag auf einer benutzerfreundlichen und intuitiven Bedienung, um die Nutzung auch für technisch unerfahrene Anwender zu vereinfachen.
\end{itemize}
