% Hier folgt der Inhalt des Abschnitts "Anforderungen und Konzept".


\subsection{\textbf{[DRAFT]}}

Im Rahmen des Projekts „Smarter Cocktailautomat“ wurden grundlegende Anforderungen definiert, die sich aus den Zielsetzungen und den zu erwartenden Einsatzszenarien ableiten lassen. Dieser Abschnitt gliedert sich in eine Analyse der funktionalen und nicht-funktionalen Anforderungen sowie die Entwicklung eines konzeptionellen Architekturansatzes, der die Interaktionen zwischen Hardware, Backend und mobiler App beschreibt.

\subsection{Anforderungsanalyse}

Die Anforderungsanalyse bildet die Grundlage für die Entwicklung des Prototyps. Dabei werden sowohl funktionale als auch nicht-funktionale Anforderungen betrachtet.

\subsection{Funktionale Anforderungen}

Funktionale Anforderungen beschreiben die Kernfunktionen, die das System erfüllen muss, um die Projektziele zu erreichen:
\begin{itemize}
	  \item Benutzerregistrierung und -verwaltung: Möglichkeit, Benutzerkonten anzulegen, zu bearbeiten und zu löschen.
	  \item Rezeptverwaltung: Hinzufügen, Bearbeiten und Löschen von Cocktailrezepten über die mobile App.
	  \item Zutatenverwaltung: Verwaltung der im Automaten verfügbaren Zutaten mit Anzeige von Füllständen.
	  \item Automatisierte Zubereitung: Präzise Dosierung und Mischung der ausgewählten Zutaten basierend auf den Rezepten.
	  \item App-Steuerung: Steuerung des Cocktailautomaten über eine intuitive Benutzeroberfläche der mobilen App.
	  \item Synchronisation mit der Cloud: Speicherung und Abruf von Benutzerdaten, Rezepten und Maschinenkonfigurationen über ein cloudbasiertes Backend.
\end{itemize}

\subsection{Nicht-funktionale Anforderungen}

Neben den funktionalen Aspekten spielen auch nicht-funktionale Anforderungen eine zentrale Rolle, um die Qualität und Benutzerfreundlichkeit des Systems sicherzustellen:

\begin{itemize}
	  \item	Skalierbarkeit: Das System muss in der Lage sein, mehrere Benutzer und Automaten zu unterstützen, ohne Leistungseinbußen zu erleiden.
	  \item	Sicherheitsaspekte: Schutz von Benutzerdaten durch verschlüsselte Kommunikation und sichere Authentifizierungsverfahren.
	  \item	Benutzerfreundlichkeit: Intuitive Bedienung der mobilen App sowie einfache Wartung und Erweiterung der Hardware.
	  \item	Fehlerresistenz: Robustheit des Systems gegenüber Hardware- und Softwarefehlern.
	  \item	Performance: Minimierung von Latenzen bei der Kommunikation zwischen App, Backend und Hardware.
\end{itemize}

\subsection{Konzeptionelle Architektur}

Auf Basis der definierten Anforderungen wurde eine konzeptionelle Architektur entwickelt, die das Gesamtsystem in seinen zentralen Komponenten beschreibt.

Gesamtsystem-Übersicht

Das System besteht aus drei Kernkomponenten:

\begin{enumerate}
  \item Hardware: Der Cocktailautomat als physisches Gerät übernimmt die Dosierung und Mischung der Zutaten.
  \item Backend: Ein containerisiertes Backend, das in Google Cloud Run gehostet wird, dient als zentrale Schnittstelle für die Speicherung und Verarbeitung von Daten.
  \item Mobile App: Eine Smartphone-App ermöglicht die Benutzerinteraktion mit dem System und bietet Funktionen wie Rezeptverwaltung und Steuerung des Automaten.
\end{enumerate}

Die Interaktionen zwischen diesen Komponenten sind in Abbildung X dargestellt. Dabei wird insbesondere der Datenfluss zwischen den Modulen hervorgehoben.

Datenflüsse und Interaktionen

\begin{itemize}
  \item Die Hardware kommuniziert mit dem Backend, um die aktuellen Maschinenzustände und Füllstände zu synchronisieren.
  \item Die mobile App greift auf das Backend zu, um Benutzerdaten, Rezepte und Maschinenkonfigurationen zu laden und zu speichern.
  \item Das Backend dient als zentrale Vermittlungsstelle und verarbeitet sowohl Steuerbefehle von der App als auch Statusmeldungen der Hardware.
\end{itemize}

\subsection{Schwerpunktsetzung}

Im Rahmen des Projekts wurde besonderes Augenmerk auf die folgenden Aspekte gelegt:

\begin{itemize}
  \item Backend: Aufgrund der hohen Anforderungen an Skalierbarkeit und Sicherheit wurde das Backend so konzipiert, dass es sowohl eine zuverlässige Kommunikation zwischen den Komponenten als auch eine effiziente Datenverwaltung ermöglicht.
  \item Mobile App: Der Fokus lag auf einer benutzerfreundlichen und intuitiven Bedienung, um die Nutzung auch für technisch unerfahrene Anwender zu vereinfachen.
  \item Hardware: Die Hardwareentwicklung zielte auf eine modulare und erweiterbare Bauweise ab, um zukünftige Anpassungen und Erweiterungen zu ermöglichen.
\end{itemize}
