\subsection{IoT und Smart-Home-Systeme}
Im Rahmen dieses Projekts definieren wir das Internet of Things (IoT) als eine Hardwareanwendung, 
die über eine Internetverbindung steuerbar ist. Der Cocktailautomat lässt sich durch eine mobile 
App aus der Ferne bedienen, wobei besonders der Cloud-Aspekt und die Integration der Steuerung über 
das Internet hervorzuheben sind. Die Wahl eines standardisierten IoT-Protokolls wie MQTT, Zigbee 
oder BLE wurde bewusst vermieden. Stattdessen wird die Kommunikation ausschließlich über das 
HTTP-Protokoll abgewickelt, was eine einfache Implementierung und Wartung ermöglicht.

\subsection{Hardware}
Für die Hardwareplattform fiel die Wahl auf den Raspberry Pi Zero 2 W. Der Pi wurde aufgrund seiner 
einfachen Programmierbarkeit und der Verfügbarkeit relevanter Python-Bibliotheken ausgewählt. 
Zunächst war auch eine Lösung zur WLAN-Konfiguration des Raspberry Pi über die App angedacht, konnte 
jedoch aufgrund technischer Schwierigkeiten nicht im zeitlichen Rahmen der Arbeit realisiert werden. 
Der Automat nutzt einen Steppermotor, einen Linearmotor, sechs Membranpumpen, zwei Mikroendschalter, 
sowie eine Waage.

\subsection{Backend und Cloud-Technologien}
Das Backend des Systems basiert auf Google Cloud Run, da es einfach aufzusetzen ist und zu Beginn 
ein kostenloses Guthaben von 300 US-Dollar bietet. Cloud Run ermöglicht die schnelle Bereitstellung 
von Containern, was die Skalierbarkeit und Wartbarkeit des Systems vereinfacht. Die 
Containerisierung des Backends sorgt für eine saubere Trennung der Anwendungslogik und die einfache 
Verwaltung der Software. Cloud SQL wurde als Datenbanklösung gewählt, da es eine hochverfügbare und 
ausfallsichere Lösung bietet und sich gut in die Google Cloud-Umgebung integriert. Als 
Programmiersprache wurde Go verwendet, ergänzt durch Bibliotheken wie \texttt{Mux}, um HTTP-Routen 
zu verwalten.

\subsection{Datenbank und API}
Für das Backend wurde eine relationale Datenbanklösung mit PostgreSQL genutzt, die aufgrund der 
guten Vorkenntnisse im Team und der Zuverlässigkeit der Technologie ausgewählt wurde. Die 
Tabellenmodelle für Cocktails und Zutaten wurden so entwickelt, dass eine effiziente und flexible 
Speicherung von Cocktailrezepten und Benutzerinformationen möglich ist. Als API-Technologie wurde 
eine REST-ähnliche Architektur gewählt, bei der statische Endpunkte definiert wurden, um Anfragen 
zu verarbeiten.

\subsection{Mobile App-Entwicklung}
Die mobile App wurde mit dem Framework \texttt{Flutter} entwickelt, da es die Möglichkeit bietet, 
für verschiedene Plattformen (iOS und Android) gleichzeitig zu entwickeln. Dies spart 
Entwicklungszeit und stellt sicher, dass die App auf mehreren Geräten konsistent funktioniert. 
Das Design der App orientiert sich an modernen ästhetischen Prinzipien, mit dem Ziel, eine 
benutzerfreundliche und ansprechende Oberfläche zu schaffen.

\subsection{Sicherheitsaspekte}
Sicherheitsmechanismen wurden sowohl im Backend als auch in der App implementiert. Wichtige 
API-Endpunkte sind durch API-Schlüssel abgesichert, wobei für die Kommunikation zwischen der App 
und der Hardware unterschiedliche Schlüssel verwendet werden. Zudem werden Endpunkte, die auf 
benutzerspezifische Daten zugreifen, mit JSON Web Tokens (JWT) geschützt, um sicherzustellen, 
dass nur autorisierte Benutzer auf ihre eigenen Daten zugreifen können. Die Kommunikation zwischen 
den Komponenten erfolgt über HTTPS, was eine sichere Übertragung der Daten gewährleistet.

\subsection{Weitere Technologien und Tools}
Für das Projekt wurde intensiv Docker und Docker Compose eingesetzt, um eine konsistente Umgebung 
für die Entwicklung und das Deployment zu schaffen. Dadurch wird sichergestellt, dass die Anwendung 
sowohl lokal als auch in der Cloud identisch läuft, was die Entwicklung und Wartung des Systems 
erheblich vereinfacht.

% \subsection{Zusammenfassung} % TODO: FIND A BETTER TITLE

% Insgesamt zeigt dieses Projekt, wie moderne Cloud- und IoT-Technologien in einem smarten System 
% kombiniert werden können, um eine benutzerfreundliche und skalierbare Lösung zu schaffen. Trotz der 
% Herausforderungen bei der Implementierung von WLAN-Konfigurationsfunktionen und der Integration 
% komplexer IoT-Protokolle konnte ein funktionierendes System entwickelt werden, das durch einfache 
% HTTP-Kommunikation zwischen den Komponenten und einer stabilen Cloud-Infrastruktur unterstützt wird.