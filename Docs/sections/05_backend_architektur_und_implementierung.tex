% Hier folgt der Inhalt des Abschnitts "Backend-Architektur und Implementierung".

\textbf{DRAFT}

Das Backend bildet das zentrale Bindeglied zwischen der mobilen App, der Hardwarekomponente des 
Cocktailautomaten und der cloudbasierten Datenbank. Es ist dafür verantwortlich, Datenflüsse zu 
koordinieren, Benutzer- und Rezeptinformationen zu verwalten und eine nahtlose Kommunikation 
zwischen den Systemkomponenten zu gewährleisten.

\subsection{Allgemeine Architektur}

Das Backend folgt einer einfachen serverbasierten Struktur und stellt HTTP-Endpunkte bereit, die 
von der mobilen App aufgerufen werden können. Diese Endpunkte aktivieren spezifische Handler-
Funktionen, die in der Regel direkte Datenbankoperationen ausführen. Zusätzlich ermöglicht das 
Backend über eine Websocket-Schnittstelle die Kommunikation mit der Hardware, um Bestellungen in 
Echtzeit an die Geräte weiterzuleiten.

Technologisch basiert das Backend vollständig auf Golang, unterstützt durch Frameworks wie gorilla/
mux für die Routenverwaltung und database/sql für den Zugriff auf die PostgreSQL-Datenbank. Die 
gesamte Umgebung ist containerisiert und nutzt Docker Compose für lokale Entwicklungsumgebungen 
sowie Google Cloud Run für das Live-Deployment. In der Live-Umgebung wird Google Cloud SQL 
anstelle einer lokalen PostgreSQL-Datenbank eingesetzt.

\subsection{Funktionale 
Anforderungen}

Das Backend übernimmt mehrere Kernfunktionen:
\begin{itemize}
	\item Benutzerdatenverwaltung: Erstellung und Authentifizierung von Benutzerkonten mittels API-
	Schlüsseln und JWT.
	\item Rezept- und Zutatenmanagement: Speicherung, Bearbeitung und Abfrage von Rezepten und deren 
	zugehörigen Zutaten.
	\item Hardwareintegration: Verwaltung der Hardwaregeräte, ihrer Zuordnung zu Benutzern und ihrer 
	Bestückung.
	\item Echtzeit-Bestellungen: Weiterleitung von Cocktailbestellungen aus der App an die 
	angeschlossene Hardware über Websockets.
\end{itemize}

Besonders anspruchsvoll ist die Echtzeitkommunikation zwischen App und Hardware. Sobald eine 
Bestellung eingeht, wird das Rezept aus der Datenbank abgerufen und über die Websocket-Verbindung 
an die entsprechende Hardware weitergeleitet. Dies ermöglicht eine nahezu sofortige Ausführung der 
Bestellung.

\subsection{Datenbank}

Für die Datenhaltung kommt PostgreSQL zum Einsatz. Die Datenbank ist in der lokalen 
Entwicklungsumgebung als Docker-Container verfügbar, während im Deployment auf Google Cloud SQL 
gesetzt wird. Zu den gespeicherten Entitäten gehören unter anderem Benutzer, Hardwaregeräte, 
Rezepte, Zutaten und deren Zuordnungen.

Das Datenbankschema wird im Anhang dokumentiert, um die Arbeit kompakt zu halten. Der 
Datenbankzugriff erfolgt mittels database/sql und direkter SQL-Statements. Durch diese Methode 
wird SQL-Injection auf Basis von Best Practices effektiv verhindert.

\subsection{API und 
Schnittstellen}

Die Backend-API ist primär REST-orientiert, verwendet jedoch keine Hypermedia-Links, da sie nicht 
für den öffentlichen Gebrauch konzipiert ist. Zu den wichtigsten Endpunkten gehören:

\textbf{Benutzerregistrierung und - login:}
\begin{enumerate}
  \item \texttt{/api/auth/register}
  \item \texttt{/api/auth/login}
\end{enumerate}

\textbf{Hardware- und Rezeptverwaltung:}
\begin{enumerate}
	\item \texttt{/api/user/hardware/\{hardware\_id\}/drinks}
	\item \texttt{/api/user/hardware/\{hardware\_id\}/recipes/\{recipe\_id\}}
	\item \texttt{/api/user/hardware/\{hardware\_id\}/slots}
\end{enumerate}

\textbf{Cocktailbestellungen:}
\begin{enumerate}
  \item \texttt{/api/user/action}
\end{enumerate}

Die Authentifizierung erfolgt durch eine Kombination aus API-Schlüsseln für allgemeine API-Zugriffe, 
Hardware-spezifischen Schlüsseln und JWT für Benutzerzugriffe .

\subsection{Cloud-Dienste und Infrastruktur}

Das Backend verwendet mehrere Dienste von Google Cloud:
\begin{itemize}
	\item Google Cloud Run: Für das Hosting der containerisierten Backend-Anwendung.
	\item Google Cloud SQL: Für die zentrale Datenhaltung.
	\item Google Artifact Registry: Zur Verwaltung von Container-Images.
	\item Cloud Logging: Zur Erfassung und Analyse von Backend- Logs.
\end{itemize}

Skalierbarkeit ist durch die Nutzung von Google Cloud Run gegeben, da die Container automatisch 
skaliert werden können. Aktuell ist diese Funktion jedoch deaktiviert, um das Gratiskontingent von 
Google nicht zu überschreiten.


\subsection{Sicherheitsaspekte}

Für die Sicherheit wurden die folgenden Maßnahmen implementiert:
\begin{itemize}
	\item HTTPS: Sämtliche Datenübertragungen erfolgen verschlüsselt.
	\item SQL-Injection-Schutz: Durch die Verwendung von database/sql wird sichergestellt, dass keine 
unsicheren SQL-Statements ausgeführt werden.
	\item API-Schlüssel und JWT: Nur authentifizierte Benutzer und Hardwaregeräte können auf die 
entsprechenden Ressourcen zugreifen.
\end{itemize}


\subsection{Implementierungsdetails}

Während der Entwicklung traten keine größeren Herausforderungen auf, abgesehen von anfänglichen 
Problemen mit dem Hosting, die durch den Wechsel zu Google Cloud Run gelöst wurden. Die 
Echtzeitkommunikation zwischen Hardware und Backend wurde erfolgreich über Websockets umgesetzt, 
um eine sofortige Reaktion auf Benutzeranfragen zu gewährleisten.


\subsection{Entwicklungsprozess}

Zur Unterstützung des Entwicklungsprozesses wurden folgende Tools und Best Practices genutzt:
\begin{itemize}
	\item Postman: Für die Erstellung und das Testen von API-Endpunkten, teils auch automatisiert.
	\item Intensives Logging: Um Debugging und Fehleranalyse zu erleichtern.
	\item Code Reviews und Code Pairing: Zur Qualitätssicherung.
	\item Agile Methoden: Planung und Umsetzung erfolgten in Sprints mithilfe eines Ticketsystems.
\end{itemize}