% Hier folgt der Inhalt des Abschnitts "Backend-Architektur und Implementierung".
Das Backend bildet das zentrale Bindeglied zwischen der mobilen App, der Hardwarekomponente des 
Cocktailautomaten und der cloudbasierten Datenbank. Es ist dafür verantwortlich, Datenflüsse zu 
koordinieren, Benutzer- und Rezeptinformationen zu verwalten und eine nahtlose Kommunikation 
zwischen den Systemkomponenten zu gewährleisten.

\subsection{Allgemeine Architektur}

Das Backend folgt einer einfachen serverbasierten Struktur und stellt HTTP - Endpunkte bereit, die 
von der mobilen App aufgerufen werden können. Diese Endpunkte aktivieren spezifische Handler-
Funktionen, die in der Regel direkte Datenbankoperationen ausführen. Zusätzlich ermöglicht das 
Backend über eine Websocket-Schnittstelle die Kommunikation mit der Hardware, um Bestellungen in 
Echtzeit an die Geräte weiterzuleiten.

Technologisch basiert das Backend vollständig auf Golang, unterstützt durch Frameworks wie \texttt{
gorilla/ mux} für die Routenverwaltung und \texttt{database/sql} für den Zugriff auf die PostgreSQL 
- Datenbank. Die gesamte Umgebung ist containerisiert und nutzt Docker Compose für lokale 
Entwicklungsumgebungen sowie Google Cloud Run für das Live-Deployment. In der Live-Umgebung wird 
Google Cloud SQL anstelle einer lokalen PostgreSQL-Datenbank eingesetzt.

\subsection{Funktionale 
Anforderungen}

Das Backend übernimmt mehrere Kernfunktionen:

\begin{itemize}
	\item \textbf{Benutzerdatenverwaltung:} Erstellung und Authentifizierung von Benutzerkonten 
	mittels API- Schlüsseln und JWT.
	\item \textbf{Rezept- und Zutatenmanagement:} Speicherung, Bearbeitung und Abfrage von Rezepten 
	und deren zugehörigen Zutaten.
	\item \textbf{Hardwareintegration:} Verwaltung der Hardwaregeräte, ihrer Zuordnung zu Benutzern 
	und ihrer Bestückung.
	\item \textbf{Echtzeit-Bestellungen:} Weiterleitung von Cocktailbestellungen aus der App an die 
	angeschlossene Hardware über Websockets.
\end{itemize}
\hyphenation{Echt-zeit-kom-mu-ni-ka-tion}
\hyphenation{Web-socket-Ver-bin-dung}
Besonders anspruchsvoll ist die Echtzeitkommunikation zwischen App und Hardware. Sobald eine 
Bestellung eingeht, wird das Rezept aus der Datenbank abgerufen und über die Websocket-Verbindung 
an die entsprechende Hardware weitergeleitet. Dies ermöglicht eine nahezu sofortige Ausführung der 
Bestellung.

\subsection{Datenbank}

Für die Datenhaltung kommt PostgreSQL zum Einsatz. Die Datenbank ist in der lokalen 
Entwicklungsumgebung als Docker-Container verfügbar, während im Deployment auf Google Cloud SQL 
gesetzt wird. Zu den gespeicherten Entitäten gehören unter anderem Benutzer, Hardwaregeräte, 
Rezepte, Zutaten und deren Zuordnungen.

Das Datenbankschema \ref{fig:database_diagram} wird im Anhang dokumentiert, um die Arbeit kompakt zu 
halten. Der Datenbankzugriff erfolgt mittels \texttt{database/sql} und direkter SQL-Statements. 
Durch diese Methode wird SQL-Injection auf Basis von Best Practices effektiv verhindert.

\subsection{API und 
Schnittstellen}

Die Backend-API ist primär REST-orientiert, verwendet jedoch keine Hypermedia-Links, da sie nicht 
für den öffentlichen Gebrauch konzipiert ist. Zu den wichtigsten Endpunkten gehören:

\vspace{0.5cm}
\textbf{Benutzerregistrierung und - login:}
\begin{enumerate}
  \item \texttt{/api/auth/register}
  \item \texttt{/api/auth/login}
\end{enumerate}

\vspace{0.5cm}
\textbf{Hardware- und Rezeptverwaltung:}
\begin{enumerate}
	\item \texttt{/api/user/hardware/\{hardware\_id\}/drinks}
	\item \texttt{/api/user/hardware/\{hardware\_id\}/recipes/\{recipe\_id\}}
	\item \texttt{/api/user/hardware/\{hardware\_id\}/slots}
\end{enumerate}

\vspace{0.5cm}
\textbf{Cocktailbestellungen:}
\begin{enumerate}
  \item \texttt{/api/user/action}
\end{enumerate}

Die Authentifizierung erfolgt durch eine Kombination aus API-Schlüsseln für allgemeine API-Zugriffe, 
Hardware-spezifischen Schlüsseln und JWT für Benutzerzugriffe .

\subsection{Cloud-Dienste und Infrastruktur}

Das Backend verwendet mehrere Dienste von Google Cloud:

\hyphenation{con-tai-ner-i-sier-ten}
\hyphenation{Back-end-An-wen-dung}
% TODO: fix hyphenation to fix overfull hbox in 'containerisierten Backend-Anwendung'
\begin{itemize}
	\item \textbf{Google Cloud Run:} Für das Hosting der containerisierten\\
	Backend-Anwendung\cite{googlecloudrun}.
	\item \textbf{Google Cloud SQL:} Für die zentrale Datenhaltung\cite{googlecloudsql}.
	\item \textbf{Google Artifact Registry:} Zur Verwaltung von Container-Images\cite{googleartifactregistry}.
	\item \textbf{Cloud Logging:} Zur Erfassung und Analyse von Backend-Logs\cite{googlecloudlogging}.
\end{itemize}
Skalierbarkeit ist durch die Nutzung von Google Cloud Run gegeben, da die Container automatisch 
skaliert werden können. Aktuell ist diese Funktion jedoch deaktiviert, um das Gratiskontingent von 
Google nicht zu überschreiten.


\subsection{Sicherheitsaspekte}

Für die Sicherheit wurden die folgenden Maßnahmen implementiert:
\begin{itemize}
	\item \textbf{HTTPS:} Sämtliche Datenübertragungen erfolgen verschlüsselt.
	\item \textbf{SQL-Injection-Schutz:} Durch die Verwendung von database/sql wird sichergestellt, 
	dass keine unsicheren SQL-Statements ausgeführt werden.
	\item \textbf{API-Schlüssel und JWT:} Nur authentifizierte Benutzer und Hardwaregeräte können auf 
	die entsprechenden Ressourcen zugreifen.
\end{itemize}


\subsection{Implementierungsdetails}

Während der Entwicklung traten keine größeren Herausforderungen auf, abgesehen von anfänglichen 
Problemen mit dem Hosting, die durch den Wechsel zu Google Cloud Run gelöst wurden. Die 
Echtzeitkommunikation zwischen Hardware und Backend wurde erfolgreich über Websockets umgesetzt, 
um eine sofortige Reaktion auf Benutzeranfragen zu gewährleisten.


\subsection{Entwicklungsprozess}

Zur Unterstützung des Entwicklungsprozesses wurden folgende Tools und Best Practices genutzt:
\begin{itemize}
	\item \textbf{Postman:} Für die Erstellung und das Testen von API-Endpunkten, teils auch automatisiert.
	\item \textbf{Intensives Logging:} Um Debugging und Fehleranalyse zu erleichtern.
	\item \textbf{Code Reviews und Code Pairing:} Zur Qualitätssicherung.
	\item \textbf{Agile Methoden:} Planung und Umsetzung erfolgten in Sprints mithilfe eines Ticketsystems.
\end{itemize}